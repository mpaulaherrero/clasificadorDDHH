\documentclass{article}%
\usepackage[T1]{fontenc}%
\usepackage[utf8]{inputenc}%
\usepackage{lmodern}%
\usepackage{textcomp}%
\usepackage{lastpage}%
\usepackage[head=40pt,margin=0.5in,bottom=0.6in]{geometry}%
\usepackage{graphicx}%
%
\title{\textbf{1.600 trabajadores de Smurfit Kappa fueron liquidados luego de intervención}}%
\author{El Nacional Web}%
\date{25/09/2018}%
%
\begin{document}%
\normalsize%
\maketitle%
\textbf{URL: }%
http://www.el{-}nacional.com/noticias/economia/1600{-}trabajadores{-}smurfit{-}kappa{-}fueron{-}liquidados{-}luego{-}intervencion\_253258\newline%
%
\textbf{Periodico: }%
EN, %
ID: %
253258, %
Seccion: %
Economía\newline%
%
\textbf{Palabras Claves: }%
Economía, Sociedad\newline%
%
\textbf{Derecho: }%
2.3, %
Otros Derechos: %
, %
Sub Derechos: %
2.3.2\newline%
%
\textbf{EP: }%
SI\newline%
\newline%
%
\textbf{\textit{El personal de la empresa se manifestó frente a una de las cuatro plantas en Carabobo, y expresaron su preocupación por la estabilidad laboral de los trabajadores en los distintos estados del país}}%
\newline%
\newline%
%
\includegraphics[width=300px]{238.jpg}%
\newline%
%
Los directivos de Smurfit Kappa Venezuela liquidaron este martes a 1600 trabajadores de todo el país luego de la intervención realizada a la empresa por parte del gobierno nacional.%
\newline%
%
El personal de la empresa se manifestó frente a una de las cuatro plantas en Carabobo, y expresaron su preocupación por la estabilidad laboral de los trabajadores en los distintos estados del país, reseñó~El Pitazo.%
\newline%
%
Marlon Mata, dirigente sindical de la planta Molinos Valencia, indicó que los trabajadores están a la expectativa, luego de recibir el pago de la liquidación sin notificación previa.%
\newline%
%
“Estamos a la expectativa, ayer al mediodía nos depositaron la liquidación de los años de servicio en la planta, yo tengo 21 años aquí y ahora quedamos así, en el aire, no sabemos que vamos a hacer, vamos a ver que nos dicen, nos preocupa la familia y con~lo que nos dieron eso no alcanza para nada, aquí estamos sobreviviendo matando tigritos, ayudante de albañilería, electricidad, mecánico, lo que me toca, yo aquí era operador.~Años atrás esto era bueno y luego decayó~la planta y mira hasta donde hemos llegado, ahora estamos afuera”, indicó el dirigente sindical de la planta Molinos Valencia, Marlon Mata.%
\newline%
%
Dany Carrera, trabajador con 10 años en la empresa, señaló que antes de la intervención la producción en las plantas de Valencia~estaba entre 50 y 40\% por debajo de su capacidad instalada.%
\newline%
%
La directiva de Smurfit Kappa Group y Smurfit Kappa Cartón de Venezuela, comunicaron a las autoridades venezolanas que, desde la notificación de la medida de ocupación temporal por parte de la Sundde, la responsabilidad plena por las operaciones de la compañía y el cumplimiento de las leyes y regulaciones aplicables~ha pasado al Estado venezolano.%
\newline%
%
Con información de~El Pitazo%
\newline%
%
\end{document}