\documentclass{article}%
\usepackage[T1]{fontenc}%
\usepackage[utf8]{inputenc}%
\usepackage{lmodern}%
\usepackage{textcomp}%
\usepackage{lastpage}%
\usepackage[head=40pt,margin=0.5in,bottom=0.6in]{geometry}%
\usepackage{graphicx}%
%
\title{\textbf{Fefarven: Escasez de medicinas llegó a 85\%}}%
\author{El Nacional Web}%
\date{19/10/2018}%
%
\begin{document}%
\normalsize%
\maketitle%
\textbf{URL: }%
http://www.el{-}nacional.com/noticias/sociedad/fefarven{-}escasez{-}medicinas{-}llego\_256407\newline%
%
\textbf{Periodico: }%
EN, %
ID: %
256407, %
Seccion: %
Sociedad\newline%
%
\textbf{Palabras Claves: }%
Salud, Escasez, Sociedad\newline%
%
\textbf{Derecho: }%
2.1, %
Otros Derechos: %
, %
Sub Derechos: %
2.1.1\newline%
%
\textbf{EP: }%
NO\newline%
\newline%
%
\textbf{\textit{El presidente de la Federación Farmacéutica Venezolana, Freddy Ceballos, señaló que la solución del desabastecimiento implicaría realizar liquidaciones especiales en divisas}}%
\newline%
\newline%
%
\includegraphics[width=300px]{262.jpg}%
\newline%
%
Freddy Ceballos, presidente de la Federación Farmacéutica Venezolana (Fefarven) señaló que el desabastecimiento de medicinas en Venezuela~llegó a 85\%.%
\newline%
%
Durante una entrevista para~Globovisión, el representante de la organización aseguró que la solución de la escasez de los medicamentos implicaría realizar liquidaciones especiales en divisas y admitir la necesidad de solucionar el desabastecimiento.%
\newline%
%
Acotó que el gobierno de Nicolás Maduro no ha aceptado que la escasez de los insumos se debe a problemas de abastecimiento, no de distribución.%
\newline%
%
Ceballos comentó que cuando las autoridades abrieron la frontera con Colombia en el estado Zulia, se permitió la entrada de medicamentos sin control alguno, lo que generó graves problemas.%
\newline%
%
Lea más en~Globovisión.%
\newline%
%
\end{document}