\documentclass{article}%
\usepackage[T1]{fontenc}%
\usepackage[utf8]{inputenc}%
\usepackage{lmodern}%
\usepackage{textcomp}%
\usepackage{lastpage}%
\usepackage[head=40pt,margin=0.5in,bottom=0.6in]{geometry}%
\usepackage{graphicx}%
%
\title{\textbf{Luisa Ortega Díaz presentará nuevas pruebas contra Maduro ante la CPI}}%
\author{El Nacional Web}%
\date{19/10/2018}%
%
\begin{document}%
\normalsize%
\maketitle%
\textbf{URL: }%
http://www.el{-}nacional.com/noticias/politica/luisa{-}ortega{-}diaz{-}presentara{-}nuevas{-}pruebas{-}contra{-}maduro{-}ante{-}cpi\_256544\newline%
%
\textbf{Periodico: }%
EN, %
ID: %
256544, %
Seccion: %
Política\newline%
%
\textbf{Palabras Claves: }%
Política, Luisa Ortega Díaz\newline%
%
\textbf{Derecho: }%
18%
, Otros Derechos: %
1.10%
, Sub Derechos: %
1.10.1.1%
\newline%
%
\textbf{EP: }%
NO\newline%
\newline%
%
\textbf{\textit{La fiscal general en el exilio~se pronunció sobre la muerte del~concejal metropolitano~Fernando Albán, y rechazó la teoría emitida desde el gobierno nacional}}%
\newline%
\newline%
%
\includegraphics[width=300px]{123.jpg}%
\newline%
%
Luisa Ortega Díaz, fiscal general de la República en el exilio, aseguró este jueves, durante una entrevista, que retornará a la Corte Penal Internacional (CPI) para consignar nuevas pruebas a la denuncia contra Nicolás Maduro.%
\newline%
%
“En los próximos días acudiré a la CPI para ampliar la denuncia contra Maduro y demostrar que existe una violación reiterada y sostenida de los derechos humanos”, indicó Ortega Díaz en una entrevista de~EVTV Miami.%
\newline%
%
La fiscal general en el exilio~se pronunció también sobre la muerte del~concejal metropolitano~Fernando Albán, y rechazó la teoría de “suicidio, ofrecida por el fiscal general designado por la asamblea nacional constituyente, Tarek William Saab.%
\newline%
%
“Este caso no es un hecho aislado, como ellos quieren hacer ver. Eso es una política sostenida del gobierno”, aseveró.%
\newline%
%
Con información de EVTV Miami%
\newline%
%
\end{document}