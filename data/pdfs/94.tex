\documentclass{article}%
\usepackage[T1]{fontenc}%
\usepackage[utf8]{inputenc}%
\usepackage{lmodern}%
\usepackage{textcomp}%
\usepackage{lastpage}%
\usepackage[head=40pt,margin=0.5in,bottom=0.6in]{geometry}%
\usepackage{graphicx}%
%
\title{\textbf{Trabajadores de la AN protestaron para exigir pago de salarios}}%
\author{El Nacional Web}%
\date{17/09/2018}%
%
\begin{document}%
\normalsize%
\maketitle%
\textbf{URL: }%
http://www.el{-}nacional.com/noticias/protestas/trabajadores{-}asamblea{-}nacional{-}protestan{-}para{-}exigir{-}pago{-}salarios\_252023\newline%
%
\textbf{Periodico: }%
EN, %
ID: %
252023, %
Seccion: %
Protestas\newline%
%
\textbf{Palabras Claves: }%
Asamblea Nacional, Economía, Protestas\newline%
%
\textbf{Derecho: }%
2.3%
, Otros Derechos: %
NO\_TIENE%
, Sub Derechos: %
2.3.4%
\newline%
%
\textbf{EP: }%
SI\newline%
\newline%
%
\textbf{\textit{Los afectados denuncian que no han recibido el 25\% de lo que corresponde a lo anunciado por el mandatario nacional}}%
\newline%
\newline%
%
\includegraphics[width=300px]{94.jpg}%
\newline%
%
Trabajadores de la Asamblea Nacional protestaron este lunes para exigir el pago de sus salarios. Denunciaron que no han recibido 25\% del salario, como lo había anunciado el presidente Nicolás Maduro.%
\newline%
%
“Hay trabajadores que no vienen porque no tienen cómo pagar el pasaje. Hemos conversado con la directiva, esta es una situación que escapa de sus manos porque la administración de recursos de la AN la~tiene el ministerio de Finanzas”, explicó~un trabajador.%
\newline%
%
Reportes de Twitter indican que los trabajadores manifestaron desde las 10:20 am en a las puertas del edificio administrativo José María Vargas.%
\newline%
%
\end{document}