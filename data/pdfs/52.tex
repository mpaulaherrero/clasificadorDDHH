\documentclass{article}%
\usepackage[T1]{fontenc}%
\usepackage[utf8]{inputenc}%
\usepackage{lmodern}%
\usepackage{textcomp}%
\usepackage{lastpage}%
\usepackage[head=40pt,margin=0.5in,bottom=0.6in]{geometry}%
\usepackage{graphicx}%
%
\title{\textbf{Espacio Público: Más de 17 personas han sido detenidas por tuitear}}%
\author{EL NACIONAL WEB}%
\date{21/10/2018}%
%
\begin{document}%
\normalsize%
\maketitle%
\textbf{URL: }%
http://www.el{-}nacional.com/noticias/sociedad/espacio{-}publico{-}mas{-}personas{-}han{-}sido{-}detenidas{-}por{-}tuitear\_256609\newline%
%
\textbf{Periodico: }%
EN, %
ID: %
256609, %
Seccion: %
Sociedad\newline%
%
\textbf{Palabras Claves: }%
NO\_TIENE\newline%
%
\textbf{Derecho: }%
1.2%
, Otros Derechos: %
17%
, Sub Derechos: %
1.2.2%
\newline%
%
\textbf{EP: }%
SI\newline%
\newline%
%
\textbf{\textit{La ONG detalló que la principal razón por la que los usuarios fueron detenidos fue por hacer comentarios sobre el gobierno nacional~}}%
\newline%
\newline%
%
\includegraphics[width=300px]{52.jpg}%
\newline%
%
Espacio Público, ONG que promueve y defiende la libertad de expresión, señaló este sábado que varios venezolanos fueron privados de libertad por expresarse en Twitter.%
\newline%
%
La ONG detalló que al menos 17 personas han sido detenidas por publicaciones en la red social. Nueves en el año 2014, una en el 2015, cinco en el 2016 y una en el 2017 y el 2018. De los detenidos 12 son hombres y cinco son mujeres.%
\newline%
%
Resaltó que entre las principales razones están: “Dirigir mensajes sobre y a distintos funcionarios del gobierno” y “criticar la situación del país en tono demandante, irónico, cuestionador y hasta retador”.%
\newline%
%
La ONG puso de ejemplo el caso de Pedro Jaimes Criollo, quien fue detenido~el 10 de mayo por publicar los datos de la ruta del avión presidencial, una información que es publica en internet.%
\newline%
%
\end{document}