\documentclass{article}%
\usepackage[T1]{fontenc}%
\usepackage[utf8]{inputenc}%
\usepackage{lmodern}%
\usepackage{textcomp}%
\usepackage{lastpage}%
\usepackage[head=40pt,margin=0.5in,bottom=0.6in]{geometry}%
\usepackage{graphicx}%
%
\title{\textbf{Ciudadanos protestaron por posible reubicación del Mercado DIno de Táchira}}%
\author{El Nacional Web}%
\date{28/09/2018}%
%
\begin{document}%
\normalsize%
\maketitle%
\textbf{URL: }%
http://www.el{-}nacional.com/noticias/sociedad/ciudadanos{-}protestaron{-}por{-}posible{-}reubicacion{-}del{-}mercado{-}dino{-}tachira\_253598\newline%
%
\textbf{Periodico: }%
EN, %
ID: %
253598, %
Seccion: %
Sociedad\newline%
%
\textbf{Palabras Claves: }%
Táchira, Protestas, Sociedad\newline%
%
\textbf{Derecho: }%
2.10, %
Otros Derechos: %
, %
Sub Derechos: %
2.10.1\newline%
%
\textbf{EP: }%
SI\newline%
\newline%
%
\textbf{\textit{Cleotilda Flores, vocera principal del mercado popular, indicó que al lugar donde los quieren pasar solo hay espacio para 200 personas y ellos son 600}}%
\newline%
\newline%
%
\includegraphics[width=300px]{250.jpg}%
\newline%
%
Trabajadores del Mercado DINO en San Cristóbal, estado Táchira, expresaron este viernes su rechazo ante la posibilidad de relocalizar el mercado en un lugar más pequeño, lo que limitaría la cantidad de negocios.%
\newline%
%
Cleotilda Flores, vocera principal del mercado popular, indicó que desde hace 28 años se encuentra en esa ubicación. También dijo que las mismas personas que laboran en el lugar son las que recogen los desperdicios que quedan cuando terminan.%
\newline%
%
“Si es tanto el problema que hagamos un referéndum y que sea todo el pueblo de San Cristóbal porque es el mercado más visitado en toda la ciudad. En menos de tres horas recogimos 1750 firmas en apoyo al mercado”, resaltó Flores durante la protesta en la alcaldía del municipio.%
\newline%
%
De la misma manera, informó que los quieren pasar a la “Villa de los buhoneros”, y que ahí solo hay capacidad para 200 personas y ellos son 600.%
\newline%
%
\end{document}