\documentclass{article}%
\usepackage[T1]{fontenc}%
\usepackage[utf8]{inputenc}%
\usepackage{lmodern}%
\usepackage{textcomp}%
\usepackage{lastpage}%
\usepackage[head=40pt,margin=0.5in,bottom=0.6in]{geometry}%
\usepackage{graphicx}%
%
\title{\textbf{Sector Salud protesta en Zulia por altos índices de desnutrición infantil}}%
\author{El Nacional Web}%
\date{20/11/2018}%
%
\begin{document}%
\normalsize%
\maketitle%
\textbf{URL: }%
http://www.el{-}nacional.com/noticias/protestas/sector{-}salud{-}protesta{-}zulia{-}por{-}altos{-}indices{-}desnutricion{-}infantil\_260417\newline%
%
\textbf{Periodico: }%
EN, %
ID: %
260417, %
Seccion: %
Protestas\newline%
%
\textbf{Palabras Claves: }%
Salud, Zulia, Protestas, Denuncia\newline%
%
\textbf{Derecho: }%
2.1, %
Otros Derechos: %
2.10, %
Sub Derechos: %
2.1.1, 2.10.1\newline%
%
\textbf{EP: }%
SI\newline%
\newline%
%
\textbf{\textit{Hania Salazar, presidenta del Colegio de Enfermeras del estado Zulia, informó que de cada 1.000 niños, 60 mueren por desnutrición}}%
\newline%
\newline%
%
\includegraphics[width=300px]{71.jpg}%
\newline%
%
Un grupo de enfermeras protesta~este martes en el estado Zulia~para denunciar el~alto índice de desnutrición en los niños por la escasez de insumos.%
\newline%
%
Hania Salazar, presidenta del~Colegio de Enfermeras de Zulia, afirmó que~al menos 60 de cada 1.000~niños en la entidad zuliana mueren por desnutrición.%
\newline%
%
Salazar agregó que es necesario un salario base de al menos 50.000 bolívares soberanos para subsistir ante la crisis del país.%
\newline%
%
El sector salud ha realizado diversas manifestaciones para denunciar la falta de medicamentos e insumos en los centros hospitalarios del país, además del incumplimiento de los contratos colectivos.%
\newline%
%
\end{document}