\documentclass{article}%
\usepackage[T1]{fontenc}%
\usepackage[utf8]{inputenc}%
\usepackage{lmodern}%
\usepackage{textcomp}%
\usepackage{lastpage}%
\usepackage[head=40pt,margin=0.5in,bottom=0.6in]{geometry}%
\usepackage{graphicx}%
%
\title{\textbf{Pensionados advierten que aumentarán colas en los bancos por pagos semanales}}%
\author{CRISBEL VARELA}%
\date{05/10/2018}%
%
\begin{document}%
\normalsize%
\maketitle%
\textbf{URL: }%
http://www.eluniversal.com/politica/22447/pensionados{-}advierten{-}que{-}aumentaran{-}colas{-}en{-}los{-}bancos{-}tras{-}pagos{-}semanales\newline%
%
\textbf{Periodico: }%
EU, %
ID: %
22447, %
Seccion: %
politica\newline%
%
\textbf{Palabras Claves: }%
NO\_TIENE\newline%
%
\textbf{Derecho: }%
2.6%
, Otros Derechos: %
NO\_TIENE%
, Sub Derechos: %
2.6.1%
\newline%
%
\textbf{EP: }%
NO\newline%
\newline%
%
\textbf{\textit{El presidente de la Federación Nacional de Jubilados y Pensionados,  Emilio Lozada, informó que el gremio se unirá a la convocatoria realizada por el Colegio de Enfermeras de Caracas}}%
\newline%
\newline%
%
\includegraphics[width=300px]{201.jpg}%
\newline%
%
Caracas.{-} El presidente de la Federación Nacional de Jubilados y Pensionados, Emilio Lozada, cuestionó el anuncio realizado este jueves por el presidente de la República, Nicolás Maduro, donde informó que se cancelará la primera semana de la quincena de octubre a los trabajadores del sector público este viernes. Además de que se continuará pagando semanalmente todo los salarios, "por lo menos hasta el mes de noviembre".%
\newline%
%
Lozada indicó que “ahora pagando semanalmente se multiplicarán las colas en los cajeros, en los bancos”.%
\newline%
%
En ese sentido, expresó que el Gobierno ha dado “un tablazo a nuestros estómagos, a nuestros salarios, a nuestra vida".%
\newline%
%
Este viernes trabajadores se concentraron en Plaza Caracas para entregar un documento a la Inspectoría del Trabajo, en rechazo a la unificación de la tabla salarial.%
\newline%
%
El presidente de la Federación Nacional de Jubilados dijo que los precios fijados por el Gobierno para los productos de la cesta básica están por encima del pago del salario mínimo contemplado en  1.800 bolívares soberanos.%
\newline%
%
“El cobro de las pensiones no es una lucha que hemos ganado, todavía falta el bono de la guerra económica, que nosotros pensamos debe ser en alguna forma resuelto, porque no es con el salario mínimo que un pensionado del seguro social puede vivir”,~aseveró Lozada.%
\newline%
%
El representante del gremio informó que también se unirán a la convocatoria realizada por el Colegio de enfermeras de Caracas fijada para el próximo 18 de este mes.%
\newline%
%
La presidenta del Colegio de Enfermeras del Distrito Capital, Ana Rosario Contreras, exigió este viernes al Ejecutivo que respete los beneficios laborales de los trabajadores del sector salud e hizo un llamado a los trabajadores a una concentración nacional en rechazo a la violación de los contratos colectivos.%
\newline%
%
\end{document}