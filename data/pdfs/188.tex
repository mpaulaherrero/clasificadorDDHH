\documentclass{article}%
\usepackage[T1]{fontenc}%
\usepackage[utf8]{inputenc}%
\usepackage{lmodern}%
\usepackage{textcomp}%
\usepackage{lastpage}%
\usepackage[head=40pt,margin=0.5in,bottom=0.6in]{geometry}%
\usepackage{graphicx}%
%
\title{\textbf{200 venezolanos detenidos en Trinidad piden retorno a Venezuela}}%
\author{El Nacional Web}%
\date{19/09/2018}%
%
\begin{document}%
\normalsize%
\maketitle%
\textbf{URL: }%
http://www.el{-}nacional.com/noticias/mundo/200{-}venezolanos{-}detenidos{-}trinidad{-}piden{-}retorno{-}venezuela\_252473\newline%
%
\textbf{Periodico: }%
EN, %
ID: %
252473, %
Seccion: %
Mundo\newline%
%
\textbf{Palabras Claves: }%
Mundo, Latinoamérica\newline%
%
\textbf{Derecho: }%
1.2, %
Otros Derechos: %
, %
Sub Derechos: %
1.2.1.3\newline%
%
\textbf{EP: }%
NO\newline%
\newline%
%
\textbf{\textit{Carlos Valero, diputado a la Asamblea Nacional, difundió un video en el que los reclusos cuestionaron al gobierno de la isla caribeña por negarles el derecho a ser extraditados al país}}%
\newline%
\newline%
%
\includegraphics[width=300px]{188.jpg}%
\newline%
%
Un grupo de venezolanos detenidos en Trinidad y Tobago solicitaron este miércoles~que el gobierno venezolano realice gestiones para repatriarlos para así poder volver a Venezuela.%
\newline%
%
Los ciudadanos fueron detenidos en junio por haber incurrido presuntamente en delitos como entrar al país de forma ilegal o en espera por deportación.%
\newline%
%
Mediante un video difundido en las redes sociales por Carlos Valero, diputado a la Asamblea Nacional , los reclusos cuestionaron al gobierno de la isla caribeña por negarles el derecho a ser extraditados a Venezuela.%
\newline%
%
"Por qué~si el gobierno de Venezuela tiene tan buenas relaciones con el gobierno de Trinidad y Tobago, y es tan generoso con este país,~nos niegan ese derecho", indicó un vocero de los detenidos en el país.%
\newline%
%
Los reclusos aseguraron que hay ciudadanos venezolanos que fueron condenados y llevados a prisión por delitos que no cometieron. Otros fueron sentenciados por entrar de forma ilegal a Trinidad, y hay presos que esperan por su proceso de deportación a Venezuela desde hace más de cinco meses.%
\newline%
%
El parlamentario Valero~aseveró~que los~venezolanos, detenidos en Trinidad y Tobago~desde junio de este año, huían de la crisis económica del país con pocos recursos económicos y por ello no pudieron cancelar multas de detención, por lo que fueron llevados a prisión.%
\newline%
%
\end{document}