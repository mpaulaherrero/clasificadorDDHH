\documentclass{article}%
\usepackage[T1]{fontenc}%
\usepackage[utf8]{inputenc}%
\usepackage{lmodern}%
\usepackage{textcomp}%
\usepackage{lastpage}%
\usepackage[head=40pt,margin=0.5in,bottom=0.6in]{geometry}%
\usepackage{graphicx}%
%
\title{\textbf{Maestros protestaron en el Ministerio de Educación este viernes}}%
\author{El Nacional Web}%
\date{21/09/2018}%
%
\begin{document}%
\normalsize%
\maketitle%
\textbf{URL: }%
http://www.el{-}nacional.com/noticias/protestas/maestros{-}protestaron{-}ministerio{-}educacion{-}este{-}viernes\_252689\newline%
%
\textbf{Periodico: }%
EN, %
ID: %
252689, %
Seccion: %
Protestas\newline%
%
\textbf{Palabras Claves: }%
Educación, Protestas, Caracas, Sociedad\newline%
%
\textbf{Derecho: }%
2.3, %
Otros Derechos: %
, %
Sub Derechos: %
2.3.4\newline%
%
\textbf{EP: }%
SI\newline%
\newline%
%
\textbf{\textit{Los ciudadanos denunciaron las condiciones en las que se encuentran las instituciones}}%
\newline%
\newline%
%
\includegraphics[width=300px]{87.jpg}%
\newline%
%
Un grupo de maestros protestó la mañana de este viernes a las afueras del Ministerio de Educación, ubicado en Caracas, por las condiciones en las que se encuentran las instituciones.%
\newline%
%
Los docentes criticaron que los planteles educativos estén en malas condiciones a una semana del inicio del periodo escolar.%
\newline%
%
Maryori Maldonado, dirigente Sindical, hizo un llamado a la sociedad civil para que se una a las protestas que realizan los distintos gremios del país.%
\newline%
%
\end{document}