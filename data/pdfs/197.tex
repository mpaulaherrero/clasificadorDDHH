\documentclass{article}%
\usepackage[T1]{fontenc}%
\usepackage[utf8]{inputenc}%
\usepackage{lmodern}%
\usepackage{textcomp}%
\usepackage{lastpage}%
\usepackage[head=40pt,margin=0.5in,bottom=0.6in]{geometry}%
\usepackage{graphicx}%
%
\title{\textbf{Denunciaron al gobierno en la OIT por eliminar los contratos}}%
\author{ANA DÍAZ | anadiaz@el{-}nacional.com}%
\date{21/09/2018}%
%
\begin{document}%
\normalsize%
\maketitle%
\textbf{URL: }%
http://www.el{-}nacional.com/noticias/economia/denunciaron{-}gobierno{-}oit{-}por{-}eliminar{-}los{-}contratos\_252640\newline%
%
\textbf{Periodico: }%
EN, %
ID: %
252640, %
Seccion: %
Economía\newline%
%
\textbf{Palabras Claves: }%
Economía, Sociedad\newline%
%
\textbf{Derecho: }%
2.3, %
Otros Derechos: %
, %
Sub Derechos: %
2.3.4\newline%
%
\textbf{EP: }%
NO\newline%
\newline%
%
\textbf{\textit{Centrales obreras solicitaron al organismo internacional su intervención inmediata ante la imposición del salario mínimo como único sueldo para los sectores público y privado}}%
\newline%
\newline%
%
\includegraphics[width=300px]{197.jpg}%
\newline%
%
La defensa del gobierno de Venezuela en su último informe a~la Organización Internacional~del Trabajo, en el que asegura~que en el país se respetan y firman los contratos colectivos, se cae por tierra con la imposición de un solo salario (el mínimo de 1.800 bolívares mensuales) que en la práctica significa eliminar la contratación colectiva.%
\newline%
%
La Unión Nacional~de Trabajadores,~la Central General~de Trabajadores y~la Confederación~de Sindicatos Autónomos de Venezuela enviaron el miércoles una comunicación al director de~la Organización Internacional~del Trabajo, Guy Ryder, en la que solicitaron~“una intervención directa e inmediata” para que el Ejecutivo venezolano cumpla los convenios 87 y 98 de~la OIT~sobre la libertad sindical y la convención colectiva.%
\newline%
%
El texto refiere el anuncio en cadena nacional del presidente Nicolás Maduro sobre su decisión unilateral de fijar el salario mínimo nacional en 1.800 bolívares soberanos mensuales a partir del 1º de septiembre y que asumirá el pago de ese sueldo o su diferencial a los trabajadores de los sectores público y privado,~a~este último durante 3 meses.%
\newline%
%
Muchas empresas se acogieron al subsidio estatal para sus nóminas, agobiadas por la crisis económica y la imposibilidad de cancelar a su personal un alza de salario de 3.364,2\% con respecto a agosto pasado, cuando el mínimo era de 30 bolívares soberanos.%
\newline%
%
Servando Carbone, coordinador de~la Unete, indicó que en la carta exponen que la unificación salarial vulnera los contratos colectivos, cuyas cláusulas de tabuladores de sueldo siempre superan al mínimo.%
\newline%
%
“El 17 de agosto Maduro decretó de manera solapada la~anulación de las convenciones colectivas de los trabajadores para pasar a una sola contratación, y echando al traste la profesionalización del trabajo”, sostuvo.%
\newline%
%
Otra violación fue aplanar las escalas salariales del sector público con diferencias muy pequeñas entre los niveles y rebajar o eliminar el pago de primas y otros beneficios laborales vigentes antes del 1º de septiembre.%
\newline%
%
Las centrales obreras, integrantes de~la Unidad~de Acción Sindical Gremial, enfatizan en la comunicación que “la clase trabajadora de Venezuela no es culpable de la crisis económica, de salud, política, social y mucho menos de una interrelación que ha venido liquidando la calidad de vida, el trabajo decente y las conquistas laborales”.%
\newline%
%
Carbone señaló que también solicitaron a~la OIT~protección especial para los dirigentes sindicales y gremiales ante las prácticas de hostigamiento y persecución del oficialismo: “Pues las protestas se elevarán de forma exponencial porque los trabajadores venezolanos no vamos a permitir que nos eliminen nuestras conquistas”.%
\newline%
%
\end{document}