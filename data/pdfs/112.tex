\documentclass{article}%
\usepackage[T1]{fontenc}%
\usepackage[utf8]{inputenc}%
\usepackage{lmodern}%
\usepackage{textcomp}%
\usepackage{lastpage}%
\usepackage[head=40pt,margin=0.5in,bottom=0.6in]{geometry}%
\usepackage{graphicx}%
%
\title{\textbf{Denuncian que el tribunal prepara un “juicio clandestino” a Requesens}}%
\author{El Nacional}%
\date{03/10/2018}%
%
\begin{document}%
\normalsize%
\maketitle%
\textbf{URL: }%
http://www.el{-}nacional.com/noticias/presos{-}politicos/denuncian{-}que{-}tribunal{-}prepara{-}juicio{-}clandestino{-}requesens\_254100\newline%
%
\textbf{Periodico: }%
EN, %
ID: %
254100, %
Seccion: %
Presos políticos\newline%
%
\textbf{Palabras Claves: }%
Presos políticos\newline%
%
\textbf{Derecho: }%
1.2, %
Otros Derechos: %
1.10, %
Sub Derechos: %
1.2.2, 1.10.1\newline%
%
\textbf{EP: }%
NO\newline%
\newline%
%
\textbf{\textit{Orianna Grannati, esposa de Requesens, aseguró que desconocen la fecha de la audiencia preliminar, que es el siguiente paso del proceso judicial. También afirmó que solo le han permitido la visita familiar una sola vez en los casi 60 días que lleva detenido el parlamentario~}}%
\newline%
\newline%
%
\includegraphics[width=300px]{112.jpg}%
\newline%
%
Al cúmulo de irregularidades y violaciones que empañan el proceso judicial que enfrenta el diputado Juan Requesens, acusado de estar implicado en el atentado contra el presidente Nicolás Maduro, se suma la “parcialidad” de la juez del Tribunal 2° de Control que no ha permitido a la defensa del parlamentario tener acceso al expediente.%
\newline%
%
Así lo denunció el abogado Joel García, quien forma parte del equipo de defensa del legislador de Primero Justicia. Indicó que a 56 días de la detención arbitraria del diputado no han tenido acceso al tribunal, ni al acta argumentativa de las acusaciones.%
\newline%
%
Señaló que el comportamiento del tribunal constituye una violación al debido proceso y al derecho a la defensa del diputado, que fue detenido sin orden judicial –violando su inmunidad parlamentaria–, desaparecido por varios días y presentado de manera extemporánea por delitos de flagrancia 10 días después del hecho del que lo señalan.%
\newline%
%
“A Requesens no se le permite tener acceso a la acusación ni al tribunal.~¿Qué tipo de proceso, qué tipo de juicio es este?~Estamos en presencia de un juicio clandestino”, expresó.%
\newline%
%
García manifestó que si no se le habilita la defensa al diputado, los abogados no pueden saber cuáles son los delitos que le imputaron y qué medios de convicción obran contra él. Todo esto constituye un obstáculo para la preparación de la defensa de Requesens.%
\newline%
%
“No sabemos por cuáles delitos lo están acusando porque no tenemos acceso a ese escrito y no lo tendremos hasta que la juez dé despacho”, expresó. Agregó que el tribunal no ha trabajado desde el 15 de agosto, cuando fue presentado Requesens.%
\newline%
%
El abogado vio con suspicacia que el viernes, cuando intentaron introducir un documento solicitando la libertad inmediata del diputado por la falta de acusación en los 45 días de investigación, fueron informados de que el tribunal había sido habilitado el jueves a las 5:00 pm para recibir la acusación contra el parlamentario. Indicó que la próxima semana emprenderán acciones en búsqueda de la justicia.%
\newline%
%
Orianna Grannati, esposa de Requesens, aseguró que desconocen la fecha de la audiencia preliminar, que es el siguiente paso del proceso judicial. También afirmó que solo le han permitido la visita familiar una sola vez en los casi 60 días que lleva detenido el parlamentario en la sede del Servicio Bolivariano de Inteligencia Nacional de El Helicoide.%
\newline%
%
\end{document}