\documentclass{article}%
\usepackage[T1]{fontenc}%
\usepackage[utf8]{inputenc}%
\usepackage{lmodern}%
\usepackage{textcomp}%
\usepackage{lastpage}%
\usepackage[head=40pt,margin=0.5in,bottom=0.6in]{geometry}%
\usepackage{graphicx}%
%
\title{\textbf{Foro Penal: 12.480 personas han sido detenidas desde 2014}}%
\author{El Nacional Web}%
\date{04/10/2018}%
%
\begin{document}%
\normalsize%
\maketitle%
\textbf{URL: }%
http://www.el{-}nacional.com/noticias/politica/foro{-}penal{-}12480{-}personas{-}han{-}sido{-}detenidas{-}desde{-}2014\_254337\newline%
%
\textbf{Periodico: }%
EN, %
ID: %
254337, %
Seccion: %
Política\newline%
%
\textbf{Palabras Claves: }%
NO\_TIENE\newline%
%
\textbf{Derecho: }%
1.2, %
Otros Derechos: %
5, 18, %
Sub Derechos: %
1.2.2\newline%
%
\textbf{EP: }%
NO\newline%
\newline%
%
\textbf{\textit{Luis Betancourt, miembro de la ONG, resaltó que 810 civiles fueron privados de libertad en tribunales militares~}}%
\newline%
\newline%
%
\includegraphics[width=300px]{161.jpg}%
\newline%
%
Luis Betancourt, miembro del Foro Penal, informó en la audiencia de la Comisión Interamericana de Derechos Humanos que desde 2014 se han realizado 12.480 detenciones arbitrarias en el país.%
\newline%
%
“Desde el año 2014 Foro Penal ha registrado 12.480 arrestos arbitrarios en todo el país, demostrando un patrón sistemático generalizado que se traduce en un aproximado de 50 personas detenidas arbitrariamente cada mes y otras 50 liberadas”, dijo Betancourt durante su ponencia.%
\newline%
%
El abogado señaló que del total de detenidos, 1.551 se convirtieron en presos políticos, 236 están presos y 7.336 continúan “sujetos a restricciones arbitrarias a su libertad a través de medidas cautelares”.%
\newline%
%
Betancourt resaltó que 810 civiles fueron privados de libertad en tribunales militares.%
\newline%
%
“Al menos 810 civiles han sido privados de libertad a través de tribunales militares y 72 aun siguen en prisión. Utilizar tribunales militares contra civiles violenta el principio el juez natural y evidencia la falta de independencia y autonomía del poder judicial”, aseguró.%
\newline%
%
\end{document}