\documentclass{article}%
\usepackage[T1]{fontenc}%
\usepackage[utf8]{inputenc}%
\usepackage{lmodern}%
\usepackage{textcomp}%
\usepackage{lastpage}%
\usepackage[head=40pt,margin=0.5in,bottom=0.6in]{geometry}%
\usepackage{graphicx}%
%
\title{\textbf{Docentes, personal obrero y administrativo de Fe y Alegría protestaron para exigir salarios justos}}%
\author{JOSÉ SILVA}%
\date{20/11/2018}%
%
\begin{document}%
\normalsize%
\maketitle%
\textbf{URL: }%
http://www.eluniversal.com/politica/26236/docentes{-}y{-}personal{-}obrero{-}de{-}fe{-}y{-}alegria{-}protestaron{-}para{-}exigir{-}salarios{-}justos\newline%
%
\textbf{Periodico: }%
EU, %
ID: %
26236, %
Seccion: %
politica\newline%
%
\textbf{Palabras Claves: }%
NO\_TIENE\newline%
%
\textbf{Derecho: }%
2.2, %
Otros Derechos: %
2.3, %
Sub Derechos: %
2.2.1, 2.3.4\newline%
%
\textbf{EP: }%
SI\newline%
\newline%
%
\textbf{\textit{Las manifestaciones se produjeron en diversos colegios de Caracas, así como en los estados Zulia, Carabobo, Aragua, Miranda y Sucre. Los profesores expresaron que su sueldo no les alcanza}}%
\newline%
\newline%
%
\includegraphics[width=300px]{110.jpg}%
\newline%
%
Caracas.{-} Profesores y personal que labora en los colegios Fe y Alegría de Venezuela protestaron la mañana de este martes en la capital y diversos estados del país para exigir mejoras salariales.%
\newline%
%
\end{document}