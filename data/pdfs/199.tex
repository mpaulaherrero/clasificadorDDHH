\documentclass{article}%
\usepackage[T1]{fontenc}%
\usepackage[utf8]{inputenc}%
\usepackage{lmodern}%
\usepackage{textcomp}%
\usepackage{lastpage}%
\usepackage[head=40pt,margin=0.5in,bottom=0.6in]{geometry}%
\usepackage{graphicx}%
%
\title{\textbf{Colombia reconoció la condición de refugiado a Julio Borges}}%
\author{El Nacional Web}%
\date{11/10/2018}%
%
\begin{document}%
\normalsize%
\maketitle%
\textbf{URL: }%
http://www.el{-}nacional.com/noticias/latinoamerica/colombia{-}reconocio{-}condicion{-}refugiado{-}julio{-}borges\_255404\newline%
%
\textbf{Periodico: }%
EN, %
ID: %
255404, %
Seccion: %
Latinoamérica\newline%
%
\textbf{Palabras Claves: }%
Política, Presos políticos\newline%
%
\textbf{Derecho: }%
1.10, %
Otros Derechos: %
CONTEXTO, %
Sub Derechos: %
1.10.1\newline%
%
\textbf{EP: }%
NO\newline%
\newline%
%
\textbf{\textit{Un canciller colombiano aprobó la solicitud presentada por Borges para ser calificado como refugiado en ese país}}%
\newline%
\newline%
%
\includegraphics[width=300px]{199.jpg}%
\newline%
%
La Cancillería de Colombia reconoció este jueves a Julio Borges, diputado en el exilio a la Asamblea Nacional, como refugiado político.%
\newline%
%
Carlos Holmes Trujillo, canciller colombiano, firmó~la resolución por la cual se reconoce la condición de refugiado en ese país~al palarmentario.%
\newline%
%
En el comunicado, el organismo indicó que se estudió la solicitud presentada por Borges y determinó que una vez analizadas las condiciones particulares y concretas de los hechos que sustentan la solicitud, "se encuentran fundados los temores de persecución aducidos por el solicitante", reza el escrito.%
\newline%
%
Borges se encuentra en el exilio luego de ser acusado de participar en el presunto atentado realizado el sábado 4 de agosto contra~ el presidente Nicolás Maduro.%
\newline%
%
\end{document}