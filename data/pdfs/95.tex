\documentclass{article}%
\usepackage[T1]{fontenc}%
\usepackage[utf8]{inputenc}%
\usepackage{lmodern}%
\usepackage{textcomp}%
\usepackage{lastpage}%
\usepackage[head=40pt,margin=0.5in,bottom=0.6in]{geometry}%
\usepackage{graphicx}%
%
\title{\textbf{Docentes de Mérida protestaron por mejora salarial}}%
\author{EL NACIONAL WEB}%
\date{13/11/2018}%
%
\begin{document}%
\normalsize%
\maketitle%
\textbf{URL: }%
http://www.el{-}nacional.com/noticias/protestas/docentes{-}merida{-}protestaron{-}por{-}mejora{-}salarial\_259560\newline%
%
\textbf{Periodico: }%
EN, %
ID: %
259560, %
Seccion: %
Protestas\newline%
%
\textbf{Palabras Claves: }%
Educación, Protestas, Sociedad\newline%
%
\textbf{Derecho: }%
2.3, %
Otros Derechos: %
, %
Sub Derechos: %
2.3.4\newline%
%
\textbf{EP: }%
SI\newline%
\newline%
%
\textbf{\textit{Profesores de la entidad se concentraron en la plaza El Llano para dirigirse hasta el Consejo Legislativo del estado de Mérida a reclamar sus derechos laborales}}%
\newline%
\newline%
%
\includegraphics[width=300px]{95.jpg}%
\newline%
%
El gremio de docentes de la ciudad de Mérida protestó~durante la mañana de este martes para exigir a las autoridades mejoras salariales y condiciones laborales óptimas.%
\newline%
%
La movilización inició en la plaza El Llano, desde donde los profesionales exigieron sus derechos. Fueron recibidos por el gobernador de la entidad Ramón Guevara.%
\newline%
%
La información fue compartida mediante Twitter por el periodista Leonardo León.%
\newline%
%
\end{document}