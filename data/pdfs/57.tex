\documentclass{article}%
\usepackage[T1]{fontenc}%
\usepackage[utf8]{inputenc}%
\usepackage{lmodern}%
\usepackage{textcomp}%
\usepackage{lastpage}%
\usepackage[head=40pt,margin=0.5in,bottom=0.6in]{geometry}%
\usepackage{graphicx}%
%
\title{\textbf{135 venezolanos están detenidos en Trinidad y Tobago}}%
\author{EL NACIONAL WEB}%
\date{24/11/2018}%
%
\begin{document}%
\normalsize%
\maketitle%
\textbf{URL: }%
http://www.el{-}nacional.com/noticias/mundo/135{-}venezolanos{-}estan{-}detenidos{-}trinidad{-}tobago\_260972\newline%
%
\textbf{Periodico: }%
EN, %
ID: %
260972, %
Seccion: %
Mundo\newline%
%
\textbf{Palabras Claves: }%
NO\_TIENE\newline%
%
\textbf{Derecho: }%
1.2, %
Otros Derechos: %
, %
Sub Derechos: %
1.2.1.3\newline%
%
\textbf{EP: }%
NO\newline%
\newline%
%
\textbf{\textit{Carlos Valero, diputado a la Asamblea Nacional, pidió que los gobiernos de Venezuela y Trindad y Tobago cooperen para solventar está situación}}%
\newline%
\newline%
%
\includegraphics[width=300px]{57.jpg}%
\newline%
%
Carlos Valero, diputado a la Asamblea Nacional, fue este viernes a un centro de detención de Inmigrantes (IDC) en Trinidad y Tobago, donde se encuentran retenidos 70 hombres y 35 mujeres venezolanas. Otros 30 están dispersos en diferentes comisarías y penales.%
\newline%
%
“Lamentable, la verdad que dan ganas de llorar. Tenemos 75 varones y aproximadamente 35 damas. El llamado para las autoridades venezolanas que están acá, por favor mayor atención consular”, dijo Valero en Twitter.%
\newline%
%
El diputado instó al gobierno de Venezuela y el de Trinidad y Tobago, que~trabajen en conjunto para repatriar a estas personas, pues el gobierno trinitense no tiene una reglamentación para atender este tipo de situaciones.%
\newline%
%
\end{document}