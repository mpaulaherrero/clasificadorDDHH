\documentclass{article}%
\usepackage[T1]{fontenc}%
\usepackage[utf8]{inputenc}%
\usepackage{lmodern}%
\usepackage{textcomp}%
\usepackage{lastpage}%
\usepackage[head=40pt,margin=0.5in,bottom=0.6in]{geometry}%
\usepackage{graphicx}%
%
\title{\textbf{CIDH solicitó al Estado venezolano informar sobre presos con tuberculosis}}%
\author{Sandra Guerrero}%
\date{28/09/2018}%
%
\begin{document}%
\normalsize%
\maketitle%
\textbf{URL: }%
http://www.el{-}nacional.com/noticias/mundo/cidh{-}solicito{-}estado{-}venezolano{-}informar{-}sobre{-}presos{-}con{-}tuberculosis\_253610\newline%
%
\textbf{Periodico: }%
EN, %
ID: %
253610, %
Seccion: %
Mundo\newline%
%
\textbf{Palabras Claves: }%
NO\_TIENE\newline%
%
\textbf{Derecho: }%
1.2%
, Otros Derechos: %
2.1, 18%
, Sub Derechos: %
1.2.4, 2.1.1%
\newline%
%
\textbf{EP: }%
NO\newline%
\newline%
%
\textbf{\textit{La situación fue manifestada por el OVP ante esa instancia, en vista de que se desconoce el número de reclusos de Uribana que padecen la enfermedad}}%
\newline%
\newline%
%
\includegraphics[width=300px]{169.jpg}%
\newline%
%
La Corte Interamericana de Derechos Humanos le solicitó al Estado venezolano un informe relacionado con el número de privados de libertad que padecen tuberculosis y que se encuentran recluidos en la cárcel de Uribana en el estado Lara.%
\newline%
%
La información la dio a conocer el Observatorio Venezolano de Prisiones que denunció la situación presentada en ese penal, que está bajo medidas provisionales.%
\newline%
%
Hasta ahora se desconoce el número exacto de presos que padecen esa enfermedad. Familiares de reclusos han informado que aproximadamente 25 detenidos se contagiaron en ese penal; agregaron que debido a la mala alimentación, el mal proliferó.%
\newline%
%
Hace más de 15 días murió un preso que lo padecía y el Estado nunca le suministró el tratamiento adecuado, y sus familiares se vieron en la necesidad de llevarle medicamentos; también se pudo conocer que algunos de estos no llegaban a las manos del paciente.%
\newline%
%
La madre de otro recluso que también padece la enfermedad se encadenó a las puertas del Palacio de Justicia, porque su hijo estaba grave y logró una medida cautelar para que lo atendieran en su casa.%
\newline%
%
El OVP le solicita al Estado venezolano que informe el número de presos que padecen la enfermedad y que se dedique a su atención.%
\newline%
%
Una fuente extraoficial informó que en algunos retenes policiales los privados de libertad padecen diversas enfermedades debido al hacinamiento, mala alimentación, falta de agua y de atención médica.%
\newline%
%
La fuente citó como ejemplo el hacinamiento en la Zona N° 7 de Boleíta, antigua sede de la PM y que hoy ocupa la PNB, cuya capacidad para albergar detenidos es de 200 personas y actualmente hay aproximadamente 800 presos que padecen diversas enfermedades, especialmente escabiosis o sarna.%
\newline%
%
\end{document}