\documentclass{article}%
\usepackage[T1]{fontenc}%
\usepackage[utf8]{inputenc}%
\usepackage{lmodern}%
\usepackage{textcomp}%
\usepackage{lastpage}%
\usepackage[head=40pt,margin=0.5in,bottom=0.6in]{geometry}%
\usepackage{graphicx}%
%
\title{\textbf{Sociedad de Infectología: Existen 4.500 casos de sarampión y 1.900 de difteria en el país}}%
\author{JOSÉ SILVA}%
\date{07/10/2018}%
%
\begin{document}%
\normalsize%
\maketitle%
\textbf{URL: }%
http://www.eluniversal.com/politica/22558/sociedad{-}de{-}infectologia{-}en{-}el{-}pais{-}existen{-}unos{-}4500{-}casos{-}de{-}sarampion{-}y{-}1900{-}de{-}difteria\newline%
%
\textbf{Periodico: }%
EU, %
ID: %
22558, %
Seccion: %
politica\newline%
%
\textbf{Palabras Claves: }%
NO\_TIENE\newline%
%
\textbf{Derecho: }%
2.1, %
Otros Derechos: %
, %
Sub Derechos: %
2.1.1\newline%
%
\textbf{EP: }%
NO\newline%
\newline%
%
\textbf{\textit{María Gacriela López, representante del organismo, señaló que desde el 2016 han fallecido 168 personas por difteria. Añadió que este año han muerto 12 infantes en el país por síndrome coqueluchoide}}%
\newline%
\newline%
%
\includegraphics[width=300px]{227.jpg}%
\newline%
%
Caracas.{-}Al menos 168 personas han muerto en todo el país desde el 2016 por casos de difteria –que se estiman en 1.900– informó María Gacriela López, presidenta de la Sociedad Venezolana de Infectología.%
\newline%
%
Señaló que la reaparición de esa enfermedad –de la que, según dijo, desde el 2016 había 24 años sin registrarse un caso en el territorio nacional– se originó en el estado Bolívar, pese a que las autoridades sanitarias tenían conocimiento de ello, y que se debió a “la disminución de las coberturas de vacunación”.%
\newline%
%
“Esto básicamente ocurre por este debilitamiento en el sistema de vigilancia en las coberturas de vacunación”, expresó la representante del organismo durante una entrevista en el programa Diálogo Con… que transmite Televen.%
\newline%
%
Argumentó que personas de todas las edades se pueden enfermar con difteria, que puede ser prevenible por vacunas, y citó el caso de un septuagenario que ingresó al Hospital Vargas de Caracas con esa enfermedad.%
\newline%
%
Por otra parte, advirtió sobre la “epidemia” del sarampión en Venezuela, que estimó en alrededor de 4.500 casos entre 2017 y 2018, de los cuales –dijo–~3.000 se produjeron este año, cifra que superó apariciones similares en los años 2001 y 2007.%
\newline%
%
“Ya tenemos circulación del virus endémicamente, es decir, constantemente en nuestros estados (…). Hay casos muy concentrados en los estados del centro: en el Distrito Capital, Miranda, Vargas, Aragua”, acotó.%
\newline%
%
López comentó que solamente en el Hospital de Niños J.M. de Los Ríos de Caracas se han registrado 1.000 casos de sarampión este año. \newline%
\newline%
“Por estas razones estamos exportando los casos a nuestros países vecinos lamentablemente”, puntualizó.%
\newline%
%
Además indicó que hasta el 31 de agosto de 2018 se habían registrado 2.160 casos de síndrome coqueluchoide “un síndrome de tos en el niño que hace recordar la tos ferina” y que, según detalló, se cobró la vida de 12 infantes~–de los cuales el 90\% era menor a seis meses–~recluidos en el Hospital de Niños.%
\newline%
%
“El caso de la tos ferina se trata con antibióticos (macrólidos)”, señaló la médico, quien recordó a su vez que en el país “hay una escasez general de todo tipo de antibiótico”.~Asimismo, refirió que “la mayoría de estos niños que han fallecido han tenido sus dosis incompletas”.%
\newline%
%
Escasez de antirretrovirales para pacientes con VIH%
\newline%
%
La doctora manifestó que los pacientes con VIH no se encuentran recibiendo, de manera permanente, el tratamiento con antirretrovirales debido a~ que durante los últimos tres años el suministro de estos medicamentos ha sido escaso, intermitente y en algunos meses "inexistente".%
\newline%
%
De acuerdo con cifras de la Sociedad Venezolana de Infectología, actualmente existen unos 2.000 niños en el país en tratamiento por tener VIH y~“por cada niño tratado probablemente hayan dos o tres sin diagnóstico", según la presidenta de la organización, quien señaló que en Venezuela no existen cifras concretas de pacientes infectados con ese~retrovirus.%
\newline%
%
Incremento de la malaria en el país y la explotación del Arco Minero%
\newline%
%
La profesional de la salud indicó que según cifras del 2017, hubo un incremento del 68\% de la malaria en el país al indicar que para mediados de ese año se registraron unos 320 mil casos con un foco importante en Bolívar que luego se distribuyó en al menos otros 17 estados.%
\newline%
%
Agregó que de acuerdo con expertos de la~Sociedad Venezolana de Infectología de Medicina Tropical las estimaciones de los casos por malaria rondan el millón, esto al incluir las recaídas y la recrudescencia de la enfermedad.%
\newline%
%
López indicó que la disponibilidad del tratamiento contra la malaria se ha visto comprometida y recordó que se trata de una "enfermedad que no se previene por vacuna, sino por control del vector. Es decir, se necesita tratamiento a los enfermos, tratamiento a las personas que van a viajar a los sitios (…) y se necesitan medidas de control ambiental y vectorial”.%
\newline%
%
Comentó que la explotación del Arco Minero, en el estado Bolívar, ha provocado que se "deterioren" las condiciones higiénico{-}sanitarias de la zona y que disminuyeron las medidas de control ambiental al igual que la vigilancia epidemiológica.%
\newline%
%
"Hay una explotación y una anarquía en relación a la explotación de oro y minas y esto ha hecho un cambio ambiental", puntualizó.%
\newline%
%
\end{document}