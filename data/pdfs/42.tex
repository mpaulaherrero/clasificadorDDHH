\documentclass{article}%
\usepackage[T1]{fontenc}%
\usepackage[utf8]{inputenc}%
\usepackage{lmodern}%
\usepackage{textcomp}%
\usepackage{lastpage}%
\usepackage[head=40pt,margin=0.5in,bottom=0.6in]{geometry}%
\usepackage{graphicx}%
%
\title{\textbf{Ocupados 21 mataderos para evitar tráfico de carne}}%
\author{MAGALY PEREZ}%
\date{01/12/2018}%
%
\begin{document}%
\normalsize%
\maketitle%
\textbf{URL: }%
http://www.eluniversal.com/economia/27169/ocupados{-}21{-}mataderos{-}para{-}evitar{-}trafico{-}de{-}carne\newline%
%
\textbf{Periodico: }%
EU, %
ID: %
27169, %
Seccion: %
economia\newline%
%
\textbf{Palabras Claves: }%
NO\_TIENE\newline%
%
\textbf{Derecho: }%
2.10, %
Otros Derechos: %
, %
Sub Derechos: %
2.10.1\newline%
%
\textbf{EP: }%
NO\newline%
\newline%
%
\textbf{\textit{Precios acordados serán publicados en Gaceta Oficial el próximo lunes}}%
\newline%
\newline%
%
\includegraphics[width=300px]{42.jpg}%
\newline%
%
Caracas.{-}El vicepresidente Tarek El Aisami anunció que se han ocupado temporalmente 21 mataderos públicos y privados a través de la Misión de Abastecimiento soberano y las gobernaciones, por un período de 180 días  para evitar el contrabando y la extracción de la carne.%
\newline%
%
“Estos mataderos han violado la política de precios acordados”, según dijo El Aisami.%
\newline%
%
Señaló que esta medida es “para evitar el bachaquerismo la extorsión y el contrabando”.%
\newline%
%
Reconoció que uno de los temas que se perturbó en las políticas implementadas en agosto fue el sector cárnico.%
\newline%
%
En ese sentido, sostuvo que “hemos establecido precios sujetos a los internacionales, y no hay excusa por parte de estos señores para haber negado el derecho a la carne de los venezolanos”, increpó.%
\newline%
%
Asimismo, se acordó prohibir la comercialización con particulares que no posean registro SICA “para golpear las mafias de corrupción”.%
\newline%
%
Agregó que con la medida también se garantiza la estabilidad laboral de los trabajadores de los mataderos.%
\newline%
%
Precios Acordados%
\newline%
%
Por otra parte, el Vicepresidente económico informó que el próximo lunes serán publicados los precios acordados de 25 productos  de la cesta básica.%
\newline%
%
Indicó que desde hace varias semanas han tenido reuniones y mesas de trabajo con representantes  del sector productivo para acordar los precios que serán publicados en la gaceta Oficial.%
\newline%
%
Los elementos que inciden en el acuerdo de los precios  son la estructura de costos, el salario, la materia prima y el transporte, explicó el funcionario.%
\newline%
%
El Aissami indicó que el presidente Maduro ha asumido la carga mayor en estos casos ya que se han otorgado divisas y créditos al sector productivo.%
\newline%
%
Agregó que las mesas de trabajo arrojaron resultados positivos y que se estará evaluando periódicamente  la estructura de costos para que los precios acordados sean justos.%
\newline%
%
Por otra parte, el alto funcionario informó que el Banco Central de Venezuela emitirá una resolución  que deroga la vigencia del bolívar fuerte, se refirió específicamente al billete de 100 Bs. F que estará en circulación hasta el miércoles 5 de diciembre.%
\newline%
%
Los ciudadanos podrán cambiar sus billetes en los bancos.%
\newline%
%
\end{document}