\documentclass{article}%
\usepackage[T1]{fontenc}%
\usepackage[utf8]{inputenc}%
\usepackage{lmodern}%
\usepackage{textcomp}%
\usepackage{lastpage}%
\usepackage[head=40pt,margin=0.5in,bottom=0.6in]{geometry}%
\usepackage{graphicx}%
%
\title{\textbf{México admite falla en atender violencia mujeres y lanza plan de protección}}%
\author{AFP}%
\date{07/03/2019}%
%
\begin{document}%
\normalsize%
\maketitle%
\textbf{URL: }%
http://www.eluniversal.com/internacional/34938/mexico{-}admite{-}falla{-}en{-}atender{-}violencia{-}mujeres{-}y{-}lanza{-}plan{-}de{-}proteccion\newline%
%
\textbf{Periodico: }%
EU, %
ID: %
34938, %
Seccion: %
internacional\newline%
%
\textbf{Palabras Claves: }%
NO\_TIENE\newline%
%
\textbf{Derecho: }%
2.1%
, Otros Derechos: %
\newline%
%
\textbf{\textit{El gobierno de Andrés Manuel López Obrador admitió que México ha fallado a la hora de proteger a las mujeres y presentó un plan para prevenir y sancionar la violencia de género}}%
\newline%
\newline%
%
\includegraphics[width=300px]{EU_34938.jpg}%
\newline%
%
Ciudad de México.{-} El gobierno de Andrés Manuel López Obrador admitió el miércoles que México ha fallado a la hora de proteger a las mujeres y presentó un plan para prevenir y sancionar la violencia de género que en 2018 se tradujo en 3.580 mujeres o niñas asesinadas.%
\newline%
%
“A todas ellas las une un factor común: la falta de intervención oportuna y diligente del Estado mexicano para preservar su integridad y para asegurar sus vidas”, reconoció la secretaria de Gobernación, Olga Sánchez Cordero, quien agregó que en los últimos años tanto la violencia común como la violencia de la delincuencia organizada se han ensañado con las mujeres, señaló AFP.%
\newline%
%
Entre las medidas del plan están identificar y corregir los vacíos y problemas estructurales de las investigaciones sobre feminicidios, violencia sexual, familiar y desapariciones; homologar los delitos contra las mujeres en los distintos estados; fortalecer la coordinación entre fiscalías, servicios de salud y autoridades para que se atienda a las mujeres “con sensibilidad y calidad” y buscar a las mujeres desde el primer aviso de desaparición, explicó Nadine Gasman, presidenta del Instituto Nacional de las Mujeres (Inmujeres), durante una conferencia de prensa junto al presidente y la secretaria de Gobernación.%
\newline%
%
Homologar el delito de feminicidio, que todavía no está tipificado en 13 estados, y cuantificarlo es una de las asignaturas pendientes de México, según han denunciado en varias ocasiones diversos colectivos de mujeres que se quejan de que sólo un buen diagnóstico podrá conllevar medidas eficaces contra este tipo de violencia.%
\newline%
%
Además el acceso a la justicia sigue pendiente en muchos casos, por ejemplo, de la fronteriza Ciudad Juárez, la localidad que en la década de los 90 se hizo tristemente famosa por los asesinatos y desapariciones de mujeres.%
\newline%
%
El Observatorio Ciudadano Nacional del Feminicidio, que agrupa a distintas organizaciones y trabaja con datos oficiales, registró más de 1.500 asesinatos de mujeres cada año desde 2014, pero menos de un tercio fueron investigados como feminicidios. Además, alertó que se trata de datos parciales ya que no todos los estados proporcionaron información. En 2017, por ejemplo, registró 1.583 en solo 18 estados.%
\newline%
%
El gobierno no especificó cuántos de los 3.580 asesinatos fueron considerados feminicidios.%
\newline%
%
Gasman, que consideró a la violencia contra las mujeres un “problema de Estado”, anunció también la creación de un registro nacional de transportes públicos (que es donde se han dado muchos casos de desapariciones de mujeres y niñas); la elaboración de un patrón de víctimas indirectas para poder atenderlas (hijos menores, por ejemplo); la detección de casos de violencia en menores a través de trabajo en las escuelas, y la puesta en marcha de aplicaciones en celulares o cualquier otro instrumento que prevenga nuevos ataques, sobre todo en las zonas de mayor incidencia.%
\newline%
%
La directora de Inmujeres reconoció que 17 estados, más de la mitad, tienen activada una alerta de género, un mecanismo que obliga a las autoridades a poner en marcha medidas adicionales ante una emergencia. Sin embargo, reconoció que esa herramienta debe mejorarse. Asimismo, se comprometió a rendir cuentas periódicamente sobre la puesta en marcha de todo el plan.%
\newline%
%
Precisamente este último punto es considerado clave para las organizaciones. Según la presidenta del Observatorio, María de la Luz Estrada, en ninguno de esos 17 estados las autoridades han evaluado si se cumplieron o no las recomendaciones. “No se están entregando resultados y la impunidad sigue creciendo”, dijo Estrada.%
\newline%
%
En concreto mencionó el caso del Estado de México, la región que rodea la capital, el más poblado del país y el primero que ha sido sentenciado por los tribunales por errores y carencias en este tema.%
\newline%
%
“No hay voluntad política, la autoridad sigue sin actuar”, lamentó y puso el ejemplo de uno de sus municipios, Ecatepec, donde no se han puesto cámaras en el río de los Remedios, un canal donde se han encontrado decenas de cuerpos desmembrados de mujeres desde hace años.%
\newline%
%
A su juicio, el gobierno federal tiene la oportunidad ahora de demostrar que quiere mejorar la situación, debe involucrarse con los estados en la desarticulación de las redes criminales que matan o desaparecen a mujeres y tiene que contar con las organizaciones civiles que tienen experiencia en el tema para definir cómo pondrá en marcha las medidas anunciadas “para que no se sigan repitiendo esquemas que no sirven”.%
\newline%
%
Sobre los refugios para las mujeres víctimas de la violencia, un tema que ha sido objeto de polémica en los últimos días, Sánchez Cordero enfatizó que seguirán funcionando pero que el gobierno actuará como rector de los mismos ya que sólo 20 de los 70 tienen parámetros de excelencia.%
\newline%
%
Declaraciones cruzadas de miembros del gobierno sugirieron que se cortaría la financiación de estos refugios y que serían las mujeres quienes recibirían ayudas puntuales en caso de ser víctimas de violencia. Sin embargo, lo que habrá será un mayor control y el dinero se destinará a los refugios a través de los estados y los municipios, aclaró la secretaria de Gobernación.%
\newline%
%
“Lo que queremos es que el Estado se ocupe de la protección a las mujeres”, subrayó López Obrador sobre los refugios.%
\newline%
%
\end{document}