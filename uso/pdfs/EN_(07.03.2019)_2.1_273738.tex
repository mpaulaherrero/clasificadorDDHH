\documentclass{article}%
\usepackage[T1]{fontenc}%
\usepackage[utf8]{inputenc}%
\usepackage{lmodern}%
\usepackage{textcomp}%
\usepackage{lastpage}%
\usepackage[head=40pt,margin=0.5in,bottom=0.6in]{geometry}%
\usepackage{graphicx}%
%
\title{\textbf{Juan Guaidó recibió en la Asamblea Nacional al embajador de Alemania}}%
\author{El Nacional}%
\date{07/03/2019}%
%
\begin{document}%
\normalsize%
\maketitle%
\textbf{URL: }%
http://www.el{-}nacional.com/noticias/politica/juan{-}guaido{-}recibio{-}asamblea{-}nacional{-}embajador{-}alemania\_273738\newline%
%
\textbf{Periodico: }%
EN, %
ID: %
273738, %
Seccion: %
Política\newline%
%
\textbf{Palabras Claves: }%
Política, Alemania, Asamblea Nacional\newline%
%
\textbf{Derecho: }%
2.1%
, Otros Derechos: %
\newline%
%
\textbf{\textit{El presidente interino indicó que la amanaza que le hicieron a Daniel Kriener no tiene cualidad}}%
\newline%
\newline%
%
\includegraphics[width=300px]{EN_273738.jpg}%
\newline%
%
Juan Guaidó, presidente interino de Venezuela, recibió este jueves en la sede de la Asamblea Nacional (AN) al embajador de Alemania,~Daniel Kriener Martín, luego de que Nicolás Maduro ordenara su expulsión de Venezuela.%
\newline%
%
"Recibimos en la Asamblea Nacional~al embajador Daniel Kriener Martín, de Alemania, a quien le manifestamos nuestro rechazo ante las amenazas del régimen usurpador", escribió Guaidó vía Twitter.%
\newline%
%
Guaidó indicó este miércoles que~Maduro pretende expulsar a quienes han apoyado con comida y medicinas a los venezolanos. "Esta amenaza verbal al embajador de Alemania no tiene ningún tipo de cualidad. Sabemos quién es el único no grato", agregó Guaidó.%
\newline%
%
\end{document}