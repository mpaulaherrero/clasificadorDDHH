\documentclass{article}%
\usepackage[T1]{fontenc}%
\usepackage[utf8]{inputenc}%
\usepackage{lmodern}%
\usepackage{textcomp}%
\usepackage{lastpage}%
\usepackage[head=40pt,margin=0.5in,bottom=0.6in]{geometry}%
\usepackage{graphicx}%
%
\title{\textbf{Guaidó: Embajada de Alemania en Venezuela está en "funcionamiento pleno"}}%
\author{JOSÉ SILVA}%
\date{07/03/2019}%
%
\begin{document}%
\normalsize%
\maketitle%
\textbf{URL: }%
http://www.eluniversal.com/politica/34997/guaido{-}embajada{-}de{-}alemania{-}en{-}venezuela{-}esta{-}en{-}funcionamiento{-}pleno\newline%
%
\textbf{Periodico: }%
EU, %
ID: %
34997, %
Seccion: %
politica\newline%
%
\textbf{Palabras Claves: }%
NO\_TIENE\newline%
%
\textbf{Derecho: }%
2.1%
, Otros Derechos: %
\newline%
%
\textbf{\textit{El presidente del Parlamento señaló que la embajada alemana en Caracas quedará bajo la autoridad de la encargada de negocios Daniela Vogl, luego de que Berlín llamara a consultas a su embajador}}%
\newline%
\newline%
%
\includegraphics[width=300px]{EU_34997.jpg}%
\newline%
%
Caracas.{-} La Embajada de Alemania en Caracas seguirá en "funcionamiento pleno" bajo la autoridad de~Daniela Vogl, encargada de negocios de ese país europeo, informó este jueves Juan Guaidó, jefe legislativo y juramentado presidente interino de Venezuela.%
\newline%
%
La decisión se da luego de que Berlín llamara a consultas al embajador alemán Daniel Kriener, a quien el Gobierno de Nicolás Maduro declaró ayer persona no grata y le dio 48 horas para que abandonara Venezuela.%
\newline%
%
A través de su cuenta en Twitter, Guaidó señaló que recibió a Kriener en la sede de la Asamblea Nacional (AN) "a quien le manifestamos nuestro rechazo ante las amenazas del régimen usurpador".%
\newline%
%
La expulsión de Kriener se produjo luego de que este recibiera a Guaidó el lunes en el Aeropuerto Internacional de Maiquetía, una postura que la Cancillería de la República consideró como "injerencia" en los asuntos internos de Venezuela.Berlín condenó la expulsión de su embajador y dijo que tal decisión "agrava" la situación en el país sudamericano. Igualmente, manifestó su reconocimiento a Guaidó.Este jueves, el ministro de Exteriores de Alemania, Heiko Maas, dijo que la presencia de diplomáticos extranjeros en el aeropuerto ayudó a impedir el arresto del jefe del Parlamento, a quien más de 50 países {-}incluida la propia Alemania{-} reconocen como presidente encargado.La Unión Europea urgió a Maduro a que reconsiderara la explusión de~Kriener.~"Lamentamos el hecho de que el embajador alemán en Venezuela se vea obligado a abandonar el país en un contexto político tenso y complejo", sostuvo en rueda de prensa la vocera de la diplomacia europea, Maja Kocijancic.%
\newline%
%
\end{document}