\documentclass{article}%
\usepackage[T1]{fontenc}%
\usepackage[utf8]{inputenc}%
\usepackage{lmodern}%
\usepackage{textcomp}%
\usepackage{lastpage}%
\usepackage[head=40pt,margin=0.5in,bottom=0.6in]{geometry}%
\usepackage{graphicx}%
%
\title{\textbf{Vitrina Venezuela}}%
\author{BENJAMIN TRIPIER}%
\date{07/03/2019}%
%
\begin{document}%
\normalsize%
\maketitle%
\textbf{URL: }%
http://www.eluniversal.com/economia/34973/vitrina{-}venezuela\newline%
%
\textbf{Periodico: }%
EU, %
ID: %
34973, %
Seccion: %
economia\newline%
%
\textbf{Palabras Claves: }%
NO\_TIENE\newline%
%
\textbf{Derecho: }%
2.1%
, Otros Derechos: %
\newline%
%
\textbf{\textit{El Estado empresario}}%
\newline%
\newline%
%
\includegraphics[width=300px]{EU_34973.jpg}%
\newline%
%
Por definición, nunca he creído en que el Estado pueda ser empresario. Pero si se insiste en hacerlo (pese a años de fracasos), habría que hacer algunos cambios, considerando que lo que es bueno para el Gobierno, no puede ser malo para la empresa; y viceversa:%
\newline%
%
■Reconocer que si se le induce toda la carga de control y fiscalización estatal, o bien la saca de mercado, o bien la obliga a tener pérdidas.%
\newline%
%
■Hay que sacarlas de la esfera de los ministerios o entes tutelares o de adscripción%
\newline%
%
■Una empresa pública debe poder quebrar; y solo permitir reposición de capital cuando las condiciones de rentabilidad y nicho de mercado a mediano plazo, lo aconsejen.%
\newline%
%
■Debe asegurarse que los nombramientos, a todos los niveles, sean basados en un esquema de carrera por mérito y resultados, y que los más altos cargos se decidan por capacidad y trayectoria gerencial.%
\newline%
%
■Debe haber una separación formal entre la generación de políticas sectoriales y la gerencia de la empresa. Esto es que la actividad se regula en su conjunto, y las empresas, las públicas y las privadas, las cumplen.%
\newline%
%
■Deben independizarse las juntas directivas con respecto a la acción de Gobierno.%
\newline%
%
■Es difícil que coexistan empresas privadas y públicas en un mismo mercado, porque si a la pública no le interesa la rentabilidad, entonces su mecanismo de toma de decisiones será distinto y creará distorsiones que en definitiva comprometerán la calidad del producto o servicio, y a la larga su continuidad.%
\newline%
%
■Si queremos un futuro mejor, atraer inversión extranjera y sin corrupción, no podemos seguir mezclando ámbitos. Esta confusión nos ha traído a este desastre en el que estamos viviendo hoy.%
\newline%
%
Noticias Destacadas%
\newline%
%
■Cerca del 90\% de la población quiere y pide a gritos un cambio%
\newline%
%
■NMM: “Los Carnavales Felices 2019 han sido un éxito total”%
\newline%
%
■Detención de Guaidó quedó en el limbo por las divisiones del chavismo%
\newline%
%
■Maduro condecora a militares que impidieron el ingreso de la ayuda humanitaria%
\newline%
%
■Trabajadores del Metro de Caracas, Corpoelec, Pdvsa, la Cancillería, el CNE entre otras instituciones gubernamentales, se reunieron con JG%
\newline%
%
■Brexit puede causar problemas con Irlanda%
\newline%
%
Lo que no es noticia (y debería serlo)%
\newline%
%
■Que la política interna se está decidiendo afuera…de lado y lado (ya no será posible que la solución venga de adentro)%
\newline%
%
■Ni que ignorar a JG, le quita al gobierno la necesidad de dar explicaciones incómodas, y a él le da capacidad de maniobra para seguir avanzando sobre lo que el chavismo llama ”espacios imaginarios”%
\newline%
%
■Que el chavismo se mueve en una realidad virtual donde hay “carnavales felices”: la gente muere en los hospitales y come de la basura%
\newline%
%
■Tampoco que las empresas van a tener que mantener (o crear, las que no las tienen) burbujas internas de bienestar para mantener la concentración de sus trabajadores, y darles todo el apoyo familiar posible…sin por eso olvidar el entorno. La RSE es más importante que nunca.%
\newline%
%
@btripier%
\newline%
%
btripier@ntn{-}consultores.com%
\newline%
%
www.ntn{-}consultores.com%
\newline%
%
\end{document}