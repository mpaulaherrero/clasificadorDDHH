\documentclass{article}%
\usepackage[T1]{fontenc}%
\usepackage[utf8]{inputenc}%
\usepackage{lmodern}%
\usepackage{textcomp}%
\usepackage{lastpage}%
\usepackage[head=40pt,margin=0.5in,bottom=0.6in]{geometry}%
%
\title{\textbf{Asamblea Nacional pide investigar muerte de indígenas pemones}}%
\author{GABRIEL SAYAGO}%
\date{07/03/2019}%
%
\begin{document}%
\normalsize%
\maketitle%
\textbf{URL: }%
http://www.eluniversal.com/politica/34959/asamblea{-}nacional{-}pide{-}investigar{-}muerte{-}de{-}indigenas{-}pemones\newline%
%
\textbf{Periodico: }%
EU, %
ID: %
34959, %
Seccion: %
politica\newline%
%
\textbf{Palabras Claves: }%
NO\_TIENE\newline%
%
\textbf{Derecho: }%
2.1%
, Otros Derechos: %
\newline%
%
\textbf{\textit{Una comisión mixta por la Asamblea Nacional elaborará informe para señalar a responsables}}%
\newline%
\newline%
%
La diputada Gladys Guaipo, presidenta de la Comisión Permanente de Pueblos Indígenas, confirmó "la muerte de cuatro pemones y tres criollos" en  Santa Elena de Uairén y manifestó que la investigación señalará a los responsables.%
\newline%
%
La parlamentaria participó en el debate de la Asamblea Nacional de este miércoles sobre la "masacre de la etnia pemón" cometida supuestamente por efectivos de seguridad.%
\newline%
%
"No tengo palabras para describir lo que presenciamos en el pueblo pemón, mi alma está herida como están heridos los corazones de los venezolanos, los indígenas fuimos los primeros habitantes del país", añadió la parlamentaria quien pidió que la Asamblea Nacional  nombre una comisión mixta para investigar la situación que se produjo a finales de febrero.%
\newline%
%
Guaipo se refirió a los hechos ocurridos en las fronteras con Colombia y Brasil para que ingresara la ayuda humanitaria. A su vez, desmintió a personas quienes acusaron a Andrés Velázquez, dirigente del partido La Causa R; y a los diputados Américo De Grazia y Luis Silva por los hechos violentos en Santa Elena de Uairén.%
\newline%
%
"Los colectivos estaban uniformados, los pemones nos decían que había hasta presos en la arremetida, en la masacre, atacados solo por pensar distinto y procurar que ingresara la ayuda humanitaria porque se están muriendo de hambre", explicó la diputada en referencia a los hechos ocurridos de los días 22 y 23 de febrero.%
\newline%
%
El diputado Américo De Grazia, condenó que Nicolás Maduro condecorara al comandante de la ZODI en el estado Bolívar, mayor general de división Alberto Bermúdez, por haber reprimido las protestas de Santa Elena de Uairén.%
\newline%
%
Asimismo, cuestionó la persecución del alcalde de la Gran Sabana. "El alcalde Emilio González, de la Gran Sabana, tiene orden de captura, y están preparando su destitución" dijo.%
\newline%
%
Queremos justicia ahora y también cuando se logre el cese de la usurpación", aseveró De Grazia en su intervención.\newline%
En otro sentido, el presidente de la Asamblea Nacional, Juan Guaidó hizo balance sobre la gira realizada  por varios países de Latinoamérica y destacó que conversó con los mandatarios sobre diversos temas.%
\newline%
%
Igualmente el presidente de la AN, hizo el anuncio de que podría atender "oportunamente", la invitación que le hiciera el Parlamento Europeo en la voz de su presidente, para asistir a una conferencia.%
\newline%
%
Exigen apertura de frontera%
\newline%
%
Este miércoles previo a la Sesión Ordinaria de la Asamblea Nacional (AN), la diputada por el estado Táchira Karim Vera, exigió ante los medios de comunicación la re{-}apertura de las zonas fronterizas entre Venezuela y Colombia. Asimismo, denunció que el cierre del tránsito hacia Colombia se ha vuelto un negocio lucrativo para funcionarios públicos.%
\newline%
%
"El día de hoy los niños de la ciudad de Ureña, acudieron nuevamente a los puentes internacionales a exigirles a la GNB, a la policía del Estado, a pedirle que les permitan el paso. Este cierre de frontera ha sido represivo, pero se ha convertido en un negocio para los servidores públicos activos" explicó Vera en rueda de prensa.%
\newline%
%
Por otra parte, el diputado Miguel Pizarró anunció que la Asamblea Nacional está trabajando en un Programa de Emergencia Social para la estabilizacIón económica. El parlamentario informó que la misma "pretende garantizar asistencia escolar, atender a las poblaciones más vulnerables, generar empleos y crear las políticas públicas necesarias para que todos tengan la oportunidad de desarrollarse".%
\newline%
%
\end{document}