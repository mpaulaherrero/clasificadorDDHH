\documentclass{article}%
\usepackage[T1]{fontenc}%
\usepackage[utf8]{inputenc}%
\usepackage{lmodern}%
\usepackage{textcomp}%
\usepackage{lastpage}%
\usepackage[head=40pt,margin=0.5in,bottom=0.6in]{geometry}%
\usepackage{graphicx}%
%
\title{\textbf{EEUU tuvo en 2018 su mayor déficit comercial en 10 años pese a los aranceles}}%
\author{AFP}%
\date{07/03/2019}%
%
\begin{document}%
\normalsize%
\maketitle%
\textbf{URL: }%
http://www.eluniversal.com/internacional/34920/eeuu{-}tuvo{-}en{-}2018{-}su{-}mayor{-}deficit{-}comercial{-}en{-}10{-}anos{-}pese{-}a{-}los{-}aranceles\newline%
%
\textbf{Periodico: }%
EU, %
ID: %
34920, %
Seccion: %
internacional\newline%
%
\textbf{Palabras Claves: }%
NO\_TIENE\newline%
%
\textbf{Derecho: }%
2.1%
, Otros Derechos: %
\newline%
%
\textbf{\textit{Los esfuerzos del presidente estadounidense, Donald Trump, por equilibrar los intercambios entre su país y el resto del mundo no surtieron efecto y el déficit comercial alcanzó en 2018}}%
\newline%
\newline%
%
\includegraphics[width=300px]{EU_34920.jpg}%
\newline%
%
Washington.{-} Los esfuerzos del presidente estadounidense, Donald Trump, por equilibrar los intercambios entre su país y el resto del mundo no surtieron efecto y el déficit comercial alcanzó en 2018 su máximo en 10 años.%
\newline%
%
Trump no pudo impedir el flujo cada vez mayor de importaciones de China, México y la Unión Europea (UE), a pesar de los aranceles adicionales que impuso a los bienes procedentes del extranjero, indicó AFP.El déficit de los bienes y servicios se situó en 621.000 millones de dólares (+12,5\%) con exportaciones récord de 2,5 billones de dólares (+6,3\%) e importaciones también históricas de 3,121 billones de dólares (+7,5\%), según los datos revelados este miércoles por el departamento de Comercio.Estados Unidos tuvo niveles de importaciones récord con 60 países; empezando con China (539.500 millones de dólares), México (346.500 millones de dólares y Alemania (125.900 millones de dólares), dice el informe.Dejando a un lado el excedente en los intercambios de servicios (+270,2\%), el déficit comercial en bienes sería de 891.300 millones de dólares, un récord absoluto.El déficit comercial estadounidense prosiguió su fuerte ascenso a pesar de la guerra comercial que la administración Trump declaró a sus principales socios, especialmente China.Washington y Pekin se impusieron mutuamente aranceles a productos que totalizan 360.000 millones de dólares. Aún así, el déficit de Estados Unidos con China se expandió a 419.000 millones de dólares; en un nuevo récord.En las últimas semanas, Trump ha señalado que Estados Unidos está cerca de cerrar su guerra comercial con China pero se conocen pocos datos concretos sobre eso.El déficit con la Unión Europea también llegó a una cifra récord: 169.300 millones de dólares mientras que con México aumentó a 81.500 millones de dólares.Estados Unidos logró superávits con Reino Unido y países de América del Sur y Centroamérica.A pesar de las medidas proteccionistas de Trump, la Casa Blanca debe lidiar con el apetito insaciable de la población por los bienes de consumo de bajo costo, procedentes del extranjero.El alza del déficit comercial no tiene por qué ser una mala noticia, ya que suele reflejar una economía en plena expansión: en 2018, el crecimiento del Producto Interior Bruto (PIB) del país rozó el 3\%.Pero los expertos coinciden en que el incremento del PIB estadounidense se frenará a medida que se reduzcan medidas de estímulo económico como la gran reducción de impuestos aprobadas por el gobierno.En busca de un acuerdo con ChinaEsos datos se publican mientras Washington y Pekín negocian la firma de un acuerdo para poner fin a su disputa. El objetivo de Estados Unidos es obtener cambios estructurales en prácticas comerciales de China que considera desleales, como las subvenciones estatales, la transferencia de tecnología impuesta por el gigante asiático o el robo de propiedad intelectual.El ministro chino de Comercio, Zhong Shan, dijo el martes que las conversaciones comerciales estaban siendo "muy difíciles" a pesar de los avances en algunos apartados, unas declaraciones similares a las que había hecho la semana pasada el representante estadounidense en las negociaciones, Robert Lighthizer.Estados Unidos también está pendiente de su relación comercial con la UE. Este miércoles, la comisaria europea de Comercio, Cecilia Malmström, se reúne con Lighthizer para proseguir el trabajo previo a unas negociaciones.La UE está bajo presión ya que Trump se plantea imponer aranceles adicionales al sector automotor, una industria clave especialmente para Alemania.El presidente estadounidense, que criticó duramente a Alemania en el pasado, recibió el 20 de febrero un informe de su ministerio de Comercio sobre la industria automotora, cuyo contenido no se ha revelado."Si no encontramos un acuerdo (comercial con la UE), impondremos aranceles" a los coches, amenazó Trump pocos días después.%
\newline%
%
\end{document}