\documentclass{article}%
\usepackage[T1]{fontenc}%
\usepackage[utf8]{inputenc}%
\usepackage{lmodern}%
\usepackage{textcomp}%
\usepackage{lastpage}%
\usepackage[head=40pt,margin=0.5in,bottom=0.6in]{geometry}%
\usepackage{graphicx}%
%
\title{\textbf{Necesaria referencia histórica}}%
\author{MIGUEL BAHACHILLE}%
\date{04/03/2019}%
%
\begin{document}%
\normalsize%
\maketitle%
\textbf{URL: }%
http://www.eluniversal.com/el{-}universal/34519/necesaria{-}referencia{-}historica\newline%
%
\textbf{Periodico: }%
EU, %
ID: %
34519, %
Seccion: %
el{-}universal\newline%
%
\textbf{Palabras Claves: }%
NO\_TIENE\newline%
%
\textbf{Derecho: }%
2.1%
, Otros Derechos: %
\newline%
%
\textbf{\textit{"El tiempo gubernativo” se usa para rendir culto y amor al eterno embalsamado en “El Cuartel de la Montaña”. Allí sí reside la predicada paz.}}%
\newline%
\newline%
%
\includegraphics[width=300px]{EU_34519.jpg}%
\newline%
%
Por primera vez en la historia de nuestro país, ahora “socialista”, se desprecia el entendimiento como dispositivo de conector social. Por la retina de todas las personas, tejido sensible a la luz, se enfocan imagines reales que se proyectan a través del cristalino. Pero el revolucionario no es así. ¡No señor! ¡En marxismo es diferente! El auténtico revolucionario está obligado a sentir, decir y ver únicamente lo que impone el líder. Por ello en Corea del Norte la gente ríe, canta, llora, come, aplaude y contrae matrimonio, cuando el presidente Kim Yong lo dispone. Igual ocurría en la extinta Unión Soviética. Los novios hacían largas colas en la Plaza Roja de Moscú esperando turno para inclinarse ante el cuerpo embalsamado de Lenin (desde 1924) y declarar ante él sus “votos de amor y esperanza”; es decir, la consumación del matrimonio.%
\newline%
%
¿A qué viene la cita?%
\newline%
%
Los crecientes apremios de más del 90\% de la población venezolana son inocultables; pero el oficialismo “no los ve”. “El tiempo gubernativo” se usa para rendir culto y amor al eterno embalsamado en “El Cuartel de la Montaña”. Allí sí reside la predicada paz. Sin embargo la constante evocación del “sempiterno adalid” no ha logrado mermar la acumulada miseria que desmerita la cualidad humana. Escasez, inflación, inseguridad, delincuencia, hambre, son reclamos del pasado. Los autores originales de la revolución venezolana asentados en “el mar de la felicidad” saben cómo “apaciguar al pueblo”. Además de la “represión educativa”, dosificar algunos melindres entre los pobres para tutelar la “paz forzada” como la de Cuba desde hace 60 años. Producción y trabajo nada interesan; ¡esas son cosas del capitalismo!%
\newline%
%
Conveniente paranoia socialista%
\newline%
%
1{-}Como las instituciones democráticas fallan, personalicemos el poder. 2{-}Como la corrupción es exclusiva del mundo civil, militaricemos todas las instancias de dominio público. 3{-}Como la diversidad hace daño al plan socialista, privilegiemos la figura del partido único. 4{-}Como las divisas provienen del mundo capitalista, asumámonos su control para evitar que se perviertan en el oscuro capitalismo. 5{-}Como la prensa no acompasa “los aciertos” del régimen revolucionario, restrinjamos la libertad de expresión y el número de medios libres. 6{-}Como los partidos de oposición han evidenciado aciertos electorales, inhabilitémoslos.%
\newline%
%
Antecedentes%
\newline%
%
Con la promulgación de la Constitución de 1999, aprobada por el 72\% de votantes con participación de un exiguo 44\%, se elimina la figura del Senado. Así, el presidente podría aprobar a su antojo los ascensos militares. Por su parte la Asamblea Nacional de entonces, con 146 de 167 escaños, aprueba en segunda discusión un proyecto de enmienda constitucional que permitiría la reelección ilimitada de cargos populares, entre ellos del presidente. Chávez insistió en alabar esa reforma pues ello constituiría “un punto de ruptura con la vieja democracia”. Ciertamente se produjo la ruptura buscada y ahora el pueblo padece las secuelas de aquel entuerto jurídico que lo tiene pasando hambre.%
\newline%
%
La realidad%
\newline%
%
Aquella Venezuela ciertamente era diferente. Hoy el país ha sido sobrepasado por la delincuencia, escasez, inseguridad, corrupción, y desempleo. Sin embargo, el régimen insiste en instituir una paranoia agrupada contra los 40 años de democracia hostigando cualquier tentativa de evocarla. Se suscita así el fanatismo que erosiona todos los derechos civiles para quebrar nuestra tradición liberal y condescender con una “autocracia popular”.%
\newline%
%
Sin embargo las masivas protestas sobre todo de los más humildes evidencian cómo el espíritu libre del venezolano no ha podido ser alienado con limosnas ni redoblando el uso de los fuetes. Ahora cada ciudadano pondera la heredad republicana iniciada en 1958. El 90\% del país rechaza la patraña socialista que insiste en someterlo por hambre. Veinte años es toda la vida para más del 60\% de los venezolanos.%
\newline%
%
miguelbmer@gmail.com%
\newline%
%
@MiguelBM29%
\newline%
%
\end{document}